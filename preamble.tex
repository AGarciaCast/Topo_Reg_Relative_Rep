
%% Conventions for todo notes:
% Informational
%% \generalExpl{Comments/directions/... in English}
\newcommand*{\generalExpl}[1]{\todo[inline]{#1}}                

% Language-specific information (currently in English or Swedish)
\newcommand*{\engExpl}[1]{\todo[inline, backgroundcolor=kth-lightgreen40]{#1}} %% \engExpl{English descriptions about formatting}
\newcommand*{\sweExpl}[1]{\todo[inline, backgroundcolor=kth-lightblue40]{#1}}  %% % \sweExpl{Text på svenska}

% warnings
\newcommand*{\warningExpl}[1]{\todo[inline, backgroundcolor=kth-lightred40]{#1}} %% \warningExpl{warnings}

% Uncomment to hide specific comments, to hide **all** ToDos add `final` to
% document class
% \renewcommand\warningExpl[1]{}
% \renewcommand\generalExpl[1]{}
% \renewcommand\engExpl[1]{}
% For example uncommenting the following line hides the Swedish language explanations
% \renewcommand\sweExpl[1]{}


% \usepackage[style=numeric,sorting=none,backend=biber]{biblatex}
\ifbiblatex
    %\usepackage[language=english,bibstyle=authoryear,citestyle=authoryear, maxbibnames=99]{biblatex}
    % alternatively you might use another style, such as IEEE and use citestyle=numeric-comp  to put multiple citations in a single pair of square brackets
    %\usepackage[style=ieee,citestyle=numeric-comp]{biblatex}
    \addbibresource{references.bib}
    %\DeclareLanguageMapping{norsk}{norwegian}
\else
    % The line(s) below are for BibTeX
    \bibliographystyle{bibstyle/myIEEEtran}
    %\bibliographystyle{apalike}
\fi


% include a variety of packages that are useful
%%%%%%%%%%%%%%%%%%%%%%%%%%%%%% Packages %%%%%%%%%%%%%%%%%%%%%%%%%%%%%%
% The following is for use with the KTH cover when not using XeLaTeX or LuaLaTeX
\ifxeorlua\relax
\else
\usepackage[scaled]{helvet}
\fi

%% The following are needed for generating the DiVA page(s)
\usepackage[force-eol=true]{scontents}              %% Needed to save lang, abstract, and keywords
\usepackage{pgffor}                 %% includes the foreach loop

%% Basic packages

%% Links
\usepackage{xurl}                %% Support for breaking URLs

%% Colorize
%\usepackage{color}
\PassOptionsToPackage{dvipsnames, svgnames, table}{xcolor}
\usepackage{xcolor}

\usepackage[normalem]{ulem}
\usepackage{soul}
\usepackage{xspace}
\usepackage{braket}

% to support units and decimal aligned columns in tables
\usepackage[locale=US]{siunitx}

\usepackage{balance}
\usepackage{stmaryrd}
\usepackage{booktabs}
\usepackage{graphicx}	        %% Support for images
\usepackage{multirow}	        %% Support for multirow columns in tables
\usepackage{tabularx}		    %% For simple table stretching
\usepackage{algorithm} 
\usepackage{algorithmic}  
\usepackage{amsfonts,amsgen,amsmath,amssymb,amsthm}
\usepackage{mathtools}
\usepackage{tensor}
\usepackage{faktor}
\usepackage{aligned-overset}

\usepackage[linesnumbered,ruled,vlined,algo2e]{algorithm2e}
% can't use both algpseudocode and algorithmic packages
%\usepackage[noend]{algpseudocode}
%\usepackage{subfig}  %% cannot use both subcaption and subfig packages
\usepackage{subcaption}
\usepackage{optidef}
\usepackage{float}		        %% Support for more flexible floating box positioning
\usepackage{pifont}

%% some additional useful packages
% to enable rotated figures
\usepackage{rotating}	    	%% For text rotating
\usepackage{array}		        %% For table wrapping
\usepackage{mdwlist}            %% various list-related commands
\usepackage{setspace}           %% For fine-grained control over line spacing


\usepackage{enumitem}           %% to allow changes to the margins of descriptions


%% If you are going to include source code (or code snippets) you can use minted in a listings environment
\usepackage{listings}		    %% For source code listing
                                %% For source code highlighting
%\usepackage[chapter, cache=false]{minted}   %% If you use minted make sure to use the chapter options to do numbering in the chapter
%%\usemintedstyle{borland}

\usepackage{bytefield}          %% For packet drawings


\setlength {\marginparwidth }{2cm} %leave some extra space for todo notes
\usepackage{todonotes}
\usepackage{notoccite} % do not number captions based on their appearance in the TOC


% Footnotes
\usepackage{perpage}
\usepackage[perpage,symbol]{footmisc} %% use symbols to ``number'' footnotes and reset which symbol is used first on each page
%% Removed option "para" to place each footnote on a separate line. This avoids bad stretching of URLs in footnotes.


%% Various useful packages
%%----------------------------------------------------------------------------
%%   pcap2tex stuff
%%----------------------------------------------------------------------------
\usepackage{tikz}
\usepackage{colortbl}
\usetikzlibrary{arrows,decorations.pathmorphing,backgrounds,fit,positioning, decorations.pathreplacing, calc,shapes}
\usepackage{pgfmath}	% --math engine
\newcommand\bmmax{2}
\usepackage{bm} % bold math


%% Managing titles
% \usepackage[outermarks]{titlesec}
%%%%%%%%%%%%%%%%%%%%%%%%%%%%%%%%%%%%%%%%%%%%%%%%%%%%%%%%%%%%%%%%%%%%%%
%\captionsetup[subfloat]{listofformat=parens}

% to include PDF pages
%\usepackage{pdfpages}

\usepackage{fvextra}
\usepackage{csquotes}               %% Recommended by biblatex
% to provide a float barrier use:
\usepackage{placeins}

\usepackage{comment}  %% Provides a comment environment
\usepackage{refcount}   %% to be able to get an expandable \getpagerefnumber

% for experiments with new cover
\usepackage{eso-pic}
\usepackage[absolute,overlay]{textpos}

% when the package is used, it draws boxes on the page showing the text, footnote, header, and margin regions of the page
%\usepackage{showframe}  
%\usepackage{printlen} % defines the printlength command to print out values of latex variable

\usepackage{xparse}  % to use for commands with optional arguments

\ifnomenclature
\usepackage[nocfg]{nomencl}  %% include refpage, refeq, to have page number and equation number for each nomenclature
\fi

\usepackage{longtable}  % For multipage tables
\usepackage{lscape}     % For landscape pages
\usepackage{needspace}  % to specify needed space, for example to keep listing heading with the listing
\usepackage{metalogo}   % for \XeLaTeX and \LuaLaTeX logos

% to define a command\B to bold font entries in a table
% based on https://tex.stackexchange.com/questions/469559/bold-entries-in-table-with-s-column-type
\usepackage{etoolbox}
\renewcommand{\bfseries}{\fontseries{b}\selectfont}
\robustify\bfseries
\newrobustcmd{\B}{\bfseries}

% To be able to have conditional text #2 that will be included IFF a label is defined, else #3
\newcommand{\iflabelexists}[3]{\ifcsundef{r@#1}{#3}{#2}}


% to allow more than 16 files to be open at once
\usepackage{morewrites}


\usepackage{tikz}
\usetikzlibrary{babel}
\usetikzlibrary{arrows}
\usetikzlibrary{automata}
\usetikzlibrary{cd}

\usepackage{subcaption}

\definecolor{greeo}{RGB}{37,211,102}
\definecolor{redp}{RGB}{255,153,153}
\usetikzlibrary{patterns}

%%% Local Variables:
%%% mode: latex
%%% TeX-master: t
%%% End:
% KTH colors for LaTeX documents
%
% Started from kthcolors by:
% Riccardo Sven Risuleo
% 2016-09-06 11:05:40
%
% from https://github.com/KTH-AC/kthcolors
%
% Adapted using the colors from "Graphic Profile Manual KTH" version 180604
% (i.e.. 2018-06-04) 
% see https://intra.kth.se/en/administration/kommunikation/grafiskprofil/kth-s-grafiska-profil-1.844676
% 
% G. Q. Maguire Jr.
% 2021-07-05
%

%\NeedsTexFormat{LaTeX2e}[1994/06/01]
%\ProvidesPackage{kthcolors}[2021/07/85 v3 Latex package with official KTH colors]

\RequirePackage{xcolor}
%% Primary colors
%% As of the new manual, there is only 1 primary color; but with three 
\definecolor{kth-blue}{RGB/cmyk}{25,84,166/0.849,0.494,0,0.349}
\colorlet{kth-blue80}{kth-blue!80!}
\colorlet{kth-blue40}{kth-blue!40!}

% these are no longer used as of 2018-06-04
%\definecolor{kth-red}{RGB/cmyk}{157,16,45/0,0.898,0.713,0.384}
%\definecolor{kth-green}{RGB/cmyk}{98,146,46/0.329,0,0.685,0.427}

%% Secondary colors
\definecolor{kth-lightblue}{RGB/cmyk}{36,160,216/0.833,0.259,0,0.153}
\colorlet{kth-lightblue80}{kth-lightblue!80!}
\colorlet{kth-lightblue40}{kth-lightblue!40!}

%\definecolor{kth-lightred}{RGB/cmyk}{228,54,62/0,0.763,0.728,0.106}
\definecolor{kth-lightred}{RGB}{216,84,151}
\colorlet{kth-lightred80}{kth-lightred!80!}
\colorlet{kth-lightred40}{kth-lightred!40!}

\definecolor{kth-lightgreen}{RGB/cmyk}{176,201,43/0.124,0,0.786,0.212} % olive
\colorlet{kth-lightgreen80}{kth-lightgreen!80!}
\colorlet{kth-lightgreen40}{kth-lightgreen!40!}

% Cool Gray 9C
%\definecolor{kth-coolgray}{RGB}{101,101,108}

% Cool Gray 10 suggested by Martin Krzywinski (see http://mkweb.bcgsc.ca/colorblind) 
\definecolor{kth-coolgray}{RGB}{99,102,106}
\colorlet{kth-coolgray80}{kth-coolgray!80!}
\colorlet{kth-coolgray40}{kth-coolgray!40!}

% Tertiary colors (yet more colors)
% All of these are no longer used
%\definecolor{kth-pink}{RGB/cmyk}{216,84,151/10,0.611,0.301,0.153}
%\definecolor{kth-yellow}{RGB/cmyk}{250,185,25/0,0.26,0.9,0.0196}
%\definecolor{kth-darkgray}{RGB/cmyk}{101,101,108/0.0648,0.0648,0,0.576}
%\definecolor{kth-middlegray}{RGB/cmyk}{189,188,188/0,0.00529,0.00529,0.259}
%\definecolor{kth-lightgray}{RGB/cmyk}{227,229,227/0.00873,0,0.00873,0.102}

%\DeclareOption{gray}{\colorlet{gray}{kth-darkgray}}

% These versions are designed to meet accessability requirements for digital media
% Note that the palette is more limited than for the print version of the colors
\ifdigitaloutput
    % primary color
    \definecolor{kth-blue}{HTML}{1954A6} % Deep sea
    \definecolor{kth-blue80}{HTML}{5E87C0}

    % Secondary colors
    \definecolor{kth-lightblue}{HTML}{2191C4} % Stratosphere
    \definecolor{kth-lightred}{HTML}{D02F80} % Fluorescence
    \definecolor{kth-lightred80}{HTML}{D95599}
    \definecolor{kth-lightgreen}{HTML}{62922E} % Front-lawn
    \definecolor{kth-coolgray}{HTML}{65656C} % Office
    \definecolor{kth-coolgray80}{HTML}{848489}
\fi



%\glsdisablehyper
%\makeglossaries
%\makenoidxglossaries
%%%% Local Variables:
%%% mode: latex
%%% TeX-master: t
%%% End:
% The following command is used with glossaries-extra
\setabbreviationstyle[acronym]{long-short}
% The form of the entries in this file is \newacronym{label}{acronym}{phrase}
%                                      or \newacronym[options]{label}{acronym}{phrase}
% see "User Manual for glossaries.sty" for the  details about the options, one example is shown below
\newacronym{ACK}{ACK}{Acknowledgement}
\newacronym{KTH}{KTH}{KTH Royal Institute of Technology}
\newacronym{NACK}{NACK}{Negative Acknowledgement}
\newacronym{UDP}{UDP}{User Datagram Protocol}
\newacronym{TDA}{TDA}{Topological Data Analysis}
\newacronym{DL}{DL}{Deep Learning}
\newacronym{UPM}{UPM}{Universidad Politéctica de Madrid}
\newacronym{CNN}{CNN}{Convolutional Neural Network}
\newacronym{ViT}{ViT}{Vision Transformer}
\newacronym{RPL}{RPL}{Division of Robotics, Perception and Learning}
\newacronym{ML}{ML}{Machine Learning}
\newacronym{CKA}{CKA}{Centered Kernel Alignment}
\newacronym{TopoML}{TopoML}{Topological Machine Learning}
\newacronym{NN}{NN}{Neural Network}
\newacronym{MAE}{MAE}{Mean Absolute Error}
\newacronym{ACC}{Acc}{Accuracy}

                %load the acronyms file

\makeatletter
\newcommand{\DeclareLatinAbbrev}[2]{%
  \DeclareRobustCommand{#1}{%
    \@ifnextchar{.}{\textit{#2}}{%
      \@ifnextchar{,}{\textit{#2.}}{%
        \@ifnextchar{!}{\textit{#2.}}{%
          \@ifnextchar{?}{\textit{#2.}}{%
            \@ifnextchar{)}{\textit{#2.}}{%
              {\textit{#2.,\ }}}}}}}}%
}
\makeatother
\DeclareLatinAbbrev{\eg}{e.g}
\DeclareLatinAbbrev{\Eg}{E.g}
\DeclareLatinAbbrev{\ie}{i.e}
\DeclareLatinAbbrev{\Ie}{I.e}
\DeclareLatinAbbrev{\etc}{etc}
\DeclareLatinAbbrev{\etal}{et~al}

\def\first {$(i)$\xspace}
\def\Second{$(ii)$\xspace}
\def\third {$(iii)$\xspace}
\def\fourth{$(iv)$\xspace}
\def\fifth {$(v)$\xspace}
\def\sixth {$(vi)$\xspace}
\def\seventh{$(vii)$\xspace}
\def\eighth{$(viii)$\xspace}

%%% custom definitions
%% Coloring the links!
\newcommand\myshade{75} % Usage: red!\myshade!black

\definecolor{ForestGreen}{RGB}{34,  139,  34}
\definecolor{HeraldRed2}{rgb}{0.81, 0.12, 0.15}

\definecolor{ffqqqq}{rgb}{1,0,0}
\definecolor{wwzzqq}{rgb}{0.4,0.6,0}
\definecolor{qqqqff}{rgb}{0,0,1}

\newcommand{\refscolor} {blue}
\newcommand{\linkscolor}{HeraldRed2}
\newcommand{\urlscolor} {ForestGreen}

%% Some definitions of used colors
%\definecolor{darkblue}{rgb}{0.0,0.0,0.3} %% define a color called darkblue
%\definecolor{darkred}{rgb}{0.4,0.0,0.0}
%\definecolor{red}{rgb}{0.7,0.0,0.0}
%\definecolor{lightgrey}{rgb}{0.8,0.8,0.8} 
%\definecolor{grey}{rgb}{0.6,0.6,0.6}
%\definecolor{darkgrey}{rgb}{0.4,0.4,0.4}
%\definecolor{aqua}{rgb}{0.0, 1.0, 1.0}

% For runin headings
\newcommand{\smartparagraph}[1]{\vspace{.05in}\noindent\textbf{#1}}

%% Table of Contents (ToC) depth 
\setcounter{secnumdepth}{4} % how many sectioning levels to assign numbers to
\setcounter{tocdepth}{4}    % how many sectioning levels to show in ToC

%% Limit hyphenation
\hyphenpenalty=9000
\tolerance=5000
% Reduce hyphenation as much as possible:
%\hyphenpenalty=15000
%\tolerance=1000

% For notes by the authors to themselves
\newcommand*{\todoinline}[1]{\textcolor{red}{TODO: #1}}

%\DeclareUnicodeCharacter{2003}{\quad}



\newcommand{\abs}[1]{\left\lvert #1 \right\rvert}

\newcommand{\norm}[1]{\left\lVert #1 \right\rVert}
\makeatletter

\newtheorem*{rep@theorem}{\rep@title}
\newcommand{\newreptheorem}[2]{%
\newenvironment{rep#1}[1]{%
 \def\rep@title{#2 \ref{##1}}%
 \begin{rep@theorem}}%
 {\end{rep@theorem}}}
\makeatother


\theoremstyle{plain}% default
\newtheorem{theorem}{Theorem}[section]
\newreptheorem{theorem}{Theorem}
\newtheorem*{corollary}{Corollary}
\newtheorem{lemma}[theorem]{Lemma}
\newreptheorem{lemma}{Lemma}
\newtheorem{proposition}{Proposition}[section]
\newreptheorem{proposition}{Proposition}

\theoremstyle{definition}
\newtheorem{definition}{Definition}[section]
\newtheorem{exmp}{Example}[section]
\newtheorem{property}{Property}[section]

\theoremstyle{remark}
\newtheorem*{remark}{Remark}
\newtheorem*{mathNote}{Note}

\newcommand\op{\mathop{\circ}\nolimits}
\newcommand{\RR}{\mathbb{R}}
\newcommand{\Anch}{\mathbb{A}}
\newcommand{\BB}{\mathbb{B}}
\newcommand{\bigO}{\mathcal{O}}
\newcommand{\cupdot}{\mathbin{\dot\cup}}
\DeclareMathOperator*{\argmax}{arg\,max}
\DeclareMathOperator*{\argmin}{arg\,min}
\newcommand\given[1][]{\:#1\vert\:}
\newcommand{\E}{\mathbb{E}}
\newcommand{\N}{\mathcal{N}}
\newcommand{\1}[2]{\mathds{1}_{#1, #2}}
\DeclarePairedDelimiter\floor{\lfloor}{\rfloor}
\newcommand{\conseq}{\overset{+}{=}}

\newcommand{\notimplies}{%
  \mathrel{{\ooalign{\hidewidth$\not\phantom{=}$\hidewidth\cr$\implies$}}}}

%command for alg-closure that automatically adapts its 'bar' to the arg based on the args length (including '\')
\newcommand{\ols}[1]{\mskip.5\thinmuskip\overline{\mskip-.5\thinmuskip {#1} \mskip-.5\thinmuskip}\mskip.5\thinmuskip} % overline short
\newcommand{\olsi}[1]{\,\overline{\!{#1}}} % overline short italic
\makeatletter
\newcommand\closure[1]{
  \tctestifnum{\count@stringtoks{#1}>1} %checks if number of chars in arg > 1 (including '\')
  {\ols{#1}} %if arg is longer than just one char, e.g. \mathbb{Q}, \mathbb{F},...
  {\olsi{#1}} %if arg is just one char, e.g. K, L,...
}
% FROM TOKCYCLE:
\long\def\count@stringtoks#1{\tc@earg\count@toks{\string#1}}
\long\def\count@toks#1{\the\numexpr-1\count@@toks#1.\tc@endcnt}
\long\def\count@@toks#1#2\tc@endcnt{+1\tc@ifempty{#2}{\relax}{\count@@toks#2\tc@endcnt}}
\def\tc@ifempty#1{\tc@testxifx{\expandafter\relax\detokenize{#1}\relax}}
\long\def\tc@earg#1#2{\expandafter#1\expandafter{#2}}
\long\def\tctestifnum#1{\tctestifcon{\ifnum#1\relax}}
\long\def\tctestifcon#1{#1\expandafter\tc@exfirst\else\expandafter\tc@exsecond\fi}
\long\def\tc@testxifx{\tc@earg\tctestifx}
\long\def\tctestifx#1{\tctestifcon{\ifx#1}}
\long\def\tc@exfirst#1#2{#1}
\long\def\tc@exsecond#1#2{#2}
\makeatother

%\renewcommand{\phi}{\varphi}
\renewcommand{\epsilon}{\varepsilon}
\newcommand{\stitch}{\mathbin{;}}


\definecolor{gray97}{gray}{.97}
\definecolor{gray75}{gray}{.75}
\definecolor{gray45}{gray}{.45}
\definecolor{lightgray}{rgb}{.9,.9,.9}

\lstset{ frame=Ltb,
     framerule=0pt,
     aboveskip=0.5cm,
     framextopmargin=3pt,
     framexbottommargin=3pt,
     framexleftmargin=0.1cm,
     framesep=0pt,
     rulesep=.4pt,
     backgroundcolor=\color{gray97},
     rulesepcolor=\color{black},
     %
     stringstyle=\ttfamily,
     showstringspaces = false,
     basicstyle=\scriptsize\ttfamily,
     commentstyle=\color{gray45},
     keywordstyle=\bfseries,
     %
     numbers=left,
     numbersep=6pt,
     numberstyle=\tiny,
     numberfirstline = false,
     breaklines=true,
     %
     extendedchars=true,
     literate=
  {á}{{\'a}}1 {é}{{\'e}}1 {í}{{\'i}}1 {ó}{{\'o}}1 {ú}{{\'u}}1
  {Á}{{\'A}}1 {É}{{\'E}}1 {Í}{{\'I}}1 {Ó}{{\'O}}1 {Ú}{{\'U}}1
  {à}{{\`a}}1 {è}{{\`e}}1 {ì}{{\`i}}1 {ò}{{\`o}}1 {ù}{{\`u}}1
  {À}{{\`A}}1 {È}{{\'E}}1 {Ì}{{\`I}}1 {Ò}{{\`O}}1 {Ù}{{\`U}}1
  {ä}{{\"a}}1 {ë}{{\"e}}1 {ï}{{\"i}}1 {ö}{{\"o}}1 {ü}{{\"u}}1
  {Ä}{{\"A}}1 {Ë}{{\"E}}1 {Ï}{{\"I}}1 {Ö}{{\"O}}1 {Ü}{{\"U}}1
  {â}{{\^a}}1 {ê}{{\^e}}1 {î}{{\^i}}1 {ô}{{\^o}}1 {û}{{\^u}}1
  {Â}{{\^A}}1 {Ê}{{\^E}}1 {Î}{{\^I}}1 {Ô}{{\^O}}1 {Û}{{\^U}}1
  {ã}{{\~a}}1 {ẽ}{{\~e}}1 {ĩ}{{\~i}}1 {õ}{{\~o}}1 {ũ}{{\~u}}1
  {Ã}{{\~A}}1 {Ẽ}{{\~E}}1 {Ĩ}{{\~I}}1 {Õ}{{\~O}}1 {Ũ}{{\~U}}1
  {œ}{{\oe}}1 {Œ}{{\OE}}1 {æ}{{\ae}}1 {Æ}{{\AE}}1 {ß}{{\ss}}1
  {ű}{{\H{u}}}1 {Ű}{{\H{U}}}1 {ő}{{\H{o}}}1 {Ő}{{\H{O}}}1
  {ç}{{\c c}}1 {Ç}{{\c C}}1 {ø}{{\o}}1 {å}{{\r a}}1 {Å}{{\r A}}1
  {€}{{\euro}}1 {£}{{\pounds}}1 {«}{{\guillemotleft}}1
  {»}{{\guillemotright}}1 {ñ}{{\~n}}1 {Ñ}{{\~N}}1 {¿}{{?`}}1 {¡}{{!`}}1 
   }
\lstnewenvironment{listing}[1][]
   {\lstset{#1}\pagebreak[0]}{\pagebreak[0]}

\lstdefinestyle{consola}{
   language=bash,
   stringstyle=\mdseries\rmfamily,
   keywordstyle=\bfseries\rmfamily,
   basicstyle=\scriptsize\bf\ttfamily,
   stepnumber=1,
   backgroundcolor=\color{lightgray},
   extendedchars=true,
   showstringspaces=false,
   showspaces=false,
   tabsize=2,
   breaklines=true,
   showtabs=false,
   captionpos=b,
   rangeprefix=-------------,
   rangesuffix=-------------,
   includerangemarker=false,
   columns=flexible,
   firstnumber=0
}


\lstdefinestyle{CodigoC}
   {basicstyle=\scriptsize,
	frame=single,
	language=C,
	numbers=left
   }
   
\lstdefinestyle{CodigoC++}
   {basicstyle=\small,
	frame=single,
	backgroundcolor=\color{gray75},
	language=C++,
	numbers=left
   }

\lstdefinestyle{Python}
   {language=Python,    
   }  % load some additional definitions to make writing more consistent

% The following is needed in conjunction with generating the DiVA data with abstracts and keywords using the scontents package and a modified listings environment
%\usepackage{listings}   %  already included
\ExplSyntaxOn
\newcommand\typestoredx[2]{\expandafter\__scontents_typestored_internal:nn\expandafter{#1} {#2}}
\ExplSyntaxOff
\makeatletter
\let\verbatimsc\@undefined
\let\endverbatimsc\@undefined
\lst@AddToHook{Init}{\hyphenpenalty=50\relax}
\makeatother


\lstnewenvironment{verbatimsc}
    {
    \lstset{%
        basicstyle=\ttfamily\tiny,
        backgroundcolor=\color{white},
        %basicstyle=\tiny,
        %columns=fullflexible,
        columns=[l]fixed,
        language=[LaTeX]TeX,
        %numbers=left,
        %numberstyle=\tiny\color{gray},
        keywordstyle=\color{red},
        breaklines=true,                 % sets automatic line breaking
        breakatwhitespace=true,          % sets if automatic breaks should only happen at whitespace
        %keepspaces=false,
        breakindent=0em,
        %fancyvrb=true,
        frame=none,                     % turn off any box
        postbreak={}                    % turn off any hook arrow for continuation lines
    }
}{}

%% Add some more keywords to bring out the structure more
\lstdefinestyle{[LaTeX]TeX}{
morekeywords={begin, todo, textbf, textit, texttt}
}

%% definition of new command for bytefield package
\newcommand{\colorbitbox}[3]{%
	\rlap{\bitbox{#2}{\color{#1}\rule{\width}{\height}}}%
	\bitbox{#2}{#3}}




% define a left aligned table cell that is ragged right
\newcolumntype{L}[1]{>{\raggedright\let\newline\\\arraybackslash\hspace{0pt}}p{#1}}

% Because backref is not compatible with biblatex
\ifbiblatex
    \usepackage[plainpages=false]{hyperref}
\else
    \usepackage[
    backref=page,
    pagebackref=false,
    plainpages=false,
                            % PDF related options
    unicode=true,           % Unicode encoded PDF strings
    bookmarks=true,         % generate bookmarks in PDF files
    bookmarksopen=false,    % Do not automatically open the bookmarks in the PDF reading program
    pdfpagemode=UseNone,    % None, UseOutlines, UseThumbs, or FullScreen
    destlabel,              % better naming of destinations
    pdfencoding=auto,       % for unicode in 
    ]{hyperref}
    \makeatletter
    \ltx@ifpackageloaded{attachfile2}{
    % cannot use backref if one is using attachfile
    }
    {\usepackage{backref}
    %
    % Customize list of backreferences.
    % From https://tex.stackexchange.com/a/183735/1340
    \renewcommand*{\backref}[1]{}
    \renewcommand*{\backrefalt}[4]{%
    \ifcase #1%
          \or [Page~#2.]%
          \else [Pages~#2.]%
    \fi%
    }
    }
    \makeatother

\fi
\usepackage[all]{hypcap}	%% prevents an issue related to hyperref and caption linking

%% Acronyms
% note that nonumberlist - removes the cross references to the pages where the acronym appears
% note that super will set the descriptions text aligned
% note that nomain - does not produce a main glossary, thus only acronyms will be in the glossary
% note that nopostdot - will prevent there being a period at the end of each entry
\usepackage[acronym, style=super, section=section, nonumberlist, nomain,
nopostdot]{glossaries}
\setlength{\glsdescwidth}{0.75\textwidth}
\usepackage[]{glossaries-extra}
\ifinswedish
    %\usepackage{glossaries-swedish}
\fi

%% For use with the README_notes
% Define a new type of glossary so that the acronyms defined in the README_notes document can be distinct from those in the thesis template
% the tlg, tld, and dn will be the file extensions used for this glossary
\newglossary[tlg]{readme}{tld}{tdn}{README acronyms}


% packages that have to be included after hyperref
\usepackage{doi}
\usepackage{cleveref}           %% Replace Section with a symbol
\usepackage{hyperxmp}           %% to be able to add the copyright information to the PDF metadata

% To be able to attach files to the final PDF
% Note that the package used is attachfile2 and not attachfile - as the former supports most TeX engines
\usepackage{attachfile2}

%% If you are going to set bidirectional text, i.e., left to right and right to left, add bidi as the last package
%\usepackage{bidi}


%\glsdisablehyper
\makeglossaries
%\makenoidxglossaries

% The following bit of ugliness is because of the problems PDFLaTeX has handling a non-breaking hyphen
% unless it is converted to UTF-8 encoding.
% If you do not use such characters in your acronyms, this could be simplified to just include the acronyms file.
\ifxeorlua
%%% Local Variables:
%%% mode: latex
%%% TeX-master: t
%%% End:
% The following command is used with glossaries-extra
\setabbreviationstyle[acronym]{long-short}
% The form of the entries in this file is \newacronym{label}{acronym}{phrase}
%                                      or \newacronym[options]{label}{acronym}{phrase}
% see "User Manual for glossaries.sty" for the  details about the options, one example is shown below
\newacronym{ACK}{ACK}{Acknowledgement}
\newacronym{KTH}{KTH}{KTH Royal Institute of Technology}
\newacronym{NACK}{NACK}{Negative Acknowledgement}
\newacronym{UDP}{UDP}{User Datagram Protocol}
\newacronym{TDA}{TDA}{Topological Data Analysis}
\newacronym{DL}{DL}{Deep Learning}
\newacronym{UPM}{UPM}{Universidad Politéctica de Madrid}
\newacronym{CNN}{CNN}{Convolutional Neural Network}
\newacronym{ViT}{ViT}{Vision Transformer}
\newacronym{RPL}{RPL}{Division of Robotics, Perception and Learning}
\newacronym{ML}{ML}{Machine Learning}
\newacronym{CKA}{CKA}{Centered Kernel Alignment}
\newacronym{TopoML}{TopoML}{Topological Machine Learning}
\newacronym{NN}{NN}{Neural Network}
\newacronym{MAE}{MAE}{Mean Absolute Error}
\newacronym{ACC}{Acc}{Accuracy}

                %load the acronyms file
\else
%%% Local Variables:
%%% mode: latex
%%% TeX-master: t
%%% End:
% The following command is used with glossaries-extra
\setabbreviationstyle[acronym]{long-short}
% The form of the entries in this file is \newacronym{label}{acronym}{phrase}
%                                      or \newacronym[options]{label}{acronym}{phrase}
% see "User Manual for glossaries.sty" for the  details about the options, one example is shown below
\newacronym{ACK}{ACK}{Acknowledgement}
\newacronym{KTH}{KTH}{KTH Royal Institute of Technology}
\newacronym{NACK}{NACK}{Negative Acknowledgement}
\newacronym{UDP}{UDP}{User Datagram Protocol}
\newacronym{TDA}{TDA}{Topological Data Analysis}
\newacronym{DL}{DL}{Deep Learning}
\newacronym{UPM}{UPM}{Universidad Politéctica de Madrid}
\newacronym{CNN}{CNN}{Convolutional Neural Network}
\newacronym{ViT}{ViT}{Vision Transformer}
\newacronym{RPL}{RPL}{Division of Robotics, Perception and Learning}
\newacronym{ML}{ML}{Machine Learning}
\newacronym{CKA}{CKA}{Centered Kernel Alignment}
\newacronym{TopoML}{TopoML}{Topological Machine Learning}


\fi



% the custom colors and the commands are defined in defines.tex    
\hypersetup{
	colorlinks  = true,
	breaklinks  = true,
	linkcolor   = \linkscolor,
	urlcolor    = \urlscolor,
	citecolor   = \refscolor,
	anchorcolor = black
}

%
% The commands below are to configure JSON listings
% 
% format for JSON listings
\colorlet{punct}{red!60!black}
\definecolor{delim}{RGB}{20,105,176}
\definecolor{numb}{RGB}{106, 109, 32}
\definecolor{string}{RGB}{0, 0, 0}

\lstdefinelanguage{json}{
    numbers=none,
    numberstyle=\small,
    frame=none,
    rulecolor=\color{black},
    showspaces=false,
    showtabs=false,
    breaklines=true,
    postbreak=\raisebox{0ex}[0ex][0ex]{\ensuremath{\color{gray}\hookrightarrow\space}},
    breakatwhitespace=true,
    basicstyle=\ttfamily\small,
    extendedchars=false,
    upquote=true,
    morestring=[b]",
    stringstyle=\color{string},
    literate=
     *{0}{{{\color{numb}0}}}{1}
      {1}{{{\color{numb}1}}}{1}
      {2}{{{\color{numb}2}}}{1}
      {3}{{{\color{numb}3}}}{1}
      {4}{{{\color{numb}4}}}{1}
      {5}{{{\color{numb}5}}}{1}
      {6}{{{\color{numb}6}}}{1}
      {7}{{{\color{numb}7}}}{1}
      {8}{{{\color{numb}8}}}{1}
      {9}{{{\color{numb}9}}}{1}
      {:}{{{\color{punct}{:}}}}{1}
      {,}{{{\color{punct}{,}}}}{1}
      {\{}{{{\color{delim}{\{}}}}{1}
      {\}}{{{\color{delim}{\}}}}}{1}
      {[}{{{\color{delim}{[}}}}{1}
      {]}{{{\color{delim}{]}}}}{1}
      {’}{{\char13}}1,
}

\lstdefinelanguage{XML}
{
  basicstyle=\ttfamily\color{blue}\bfseries\small,
  morestring=[b]",
  morestring=[s]{>}{<},
  morecomment=[s]{<?}{?>},
  stringstyle=\color{black},
  identifierstyle=\color{blue},
  keywordstyle=\color{cyan},
  breaklines=true,
  postbreak=\raisebox{0ex}[0ex][0ex]{\ensuremath{\color{gray}\hookrightarrow\space}},
  breakatwhitespace=true,
  morekeywords={xmlns,version,type}% list your attributes here
}

% In case you use both listings and lstlistings - this makes them both use the same counter
\makeatletter
\AtBeginDocument{\let\c@listing\c@lstlisting}
\makeatother
\usepackage{subfiles}

% To have Creative Commons (CC) license and logos use the doclicense package
% Note that the lowercase version of the license has to be used in the modifier
% i.e., one of by, by-nc, by-nd, by-nc-nd, by-sa, by-nc-sa, zero.
% For background see:
% https://www.kb.se/samverkan-och-utveckling/oppen-tillgang-och-bibsamkonsortiet/open-access-and-bibsam-consortium/open-access/creative-commons-faq-for-researchers.html
% https://kib.ki.se/en/publish-analyse/publish-your-article-open-access/open-licence-your-publication-cc
\begin{comment}
\usepackage[
    type={CC},
    %modifier={by-nc-nd},
    %version={4.0},
    modifier={by-nc},
    imagemodifier={-eu-88x31},  % to get Euro symbol rather than Dollar sign
    hyphenation={RaggedRight},
    version={4.0},
    %modifier={zero},
    %version={1.0},
]{doclicense}
\end{comment}