
\documentclass[../main.tex]{subfiles}
\begin{document}

\begin{abstract}
% The first abstract should be in the language of the thesis.
% Abstract fungerar på svenska också.
\markboth{\abstractname}{}
\begin{scontents}[store-env=lang]
eng
\end{scontents}
%%% The contents of the abstract (between the begin and end of scontents) will be saved in LaTeX format
%%% and output on the page(s) at the end of the thesis with information for DiVA facilitating the correct
%%% entry of the meta data for your thesis.
%%% These page(s) will be removed before the thesis is inserted into DiVA.

\begin{scontents}[store-env=abstracts,print-env=true]
This Master's Thesis delves into the application of topological regularization techniques and relative latent representations within the realm of zero-shot model stitching. Building upon the prior work of Moschella et al. (2022) that introduces relative latent representations to enhance the similarities between latent spaces of different models, we incorporate the approach of Hofer et al. (2021), which combines Topological Data Analysis (TDA) and Machine Learning techniques for topological densification of class distributions in the latent space.

The main research objective is to investigate the impact of topological regularization on zero-shot stitching performance when employing relative latent representations. Theoretical foundations for the relative transformation are established based on the intertwiner groups of activation functions. Empirical analyses are conducted to validate the assumptions underlying the construction of the relative transformation in the latent space. Moreover, experiments are performed on a Large Language Model trained on multilingual Amazon Reviews datasets to evaluate the effectiveness of zero-shot stitching while using the topological densification technique and the relative transformation.

The findings indicate that the proposed methodologies can enhance the performance of multilingual model stitching. Specifically, enforcing the relative transformation to preserve the H\textsubscript{0} homology death times distributions proves beneficial. Additionally, the presence of similar topological features plays a crucial role in achieving higher model compatibility. However, a more in-depth exploration of the geometric properties of the post-relative transformation latent space is necessary to further improve the topological densification technique.

Overall, this work contributes to the emerging field of Topological Machine Learning and provides valuable insights for researchers in transfer learning and representation learning domains.
\end{scontents}


\subsection*{Keywords}
\begin{scontents}[store-env=keywords,print-env=true]
% If you set the EnglishKeywords earlier, you can retrieve them with:
\InsertKeywords{english}
% If you did not set the EnglishKeywords earlier then simply enter the keywords here:
% comma separate keywords, such as: Canvas Learning Management System, Docker containers, Performance tuning
\end{scontents}
\end{abstract}

\cleardoublepage
\babelpolyLangStart{swedish}
\begin{abstract}
    \markboth{\abstractname}{}
\begin{scontents}[store-env=lang]
swe
\end{scontents}

\begin{scontents}[store-env=abstracts,print-env=true]
Denna masteruppsats undersöker tillämpningen av topologiska regleringstekniker och relativa latenta representationer inom området för zero-shot model stitching. Genom att bygga vidare på tidigare arbete av Moschella et al. (2022), som introducerade relativa latenta representationer för att förbättra likheterna mellan latenta rummet hos olika modeller, inkorporerar vi tillvägagångssättet av Hofer et al. (2021), som kombinerar topologisk dataanalys (TDA) och maskininlärningstekniker för topologisk ``förtätning'' av klassfördelningar i det latenta utrymmet.

Den huvudsakliga forskningsuppgiften är att undersöka effekten av topologisk reglering på zero-shot model stitching-prestanda när man använder relativa latenta representationer. Teoretiska grunder för den relativa transformationen etableras baserat på intertwinergrupperna för aktiveringsfunktioner. Empiriska analyser genomförs för att validera antagandena som ligger till grund för konstruktionen av den relativa transformationen i det latenta rummen. Dessutom utförs experiment på en stor språkmodell tränad på multilinguella Amazon Reviews-dataset för att utvärdera effektiviteten hos zero-shot model stitching med Hofer's topologiska reglering och relativa transformation.

Resultaten visar att de föreslagna metoderna kan förbättra prestationen hos zero-shot model stitching för flerspråkiga modeller. Specifikt är det fördelaktigt att tvinga den relativa transformationen att bevara H\textsubscript{0} homologins dödstidsfördelningar. Dessutom spelar närvaron av liknande topologiska egenskaper en avgörande roll för att uppnå högre modellkompatibilitet. Dock krävs en mer ingående utforskning av de geometriska egenskaperna hos det latenta utrymmet efter den relativa transformationen för att ytterligare förbättra Hofer's topologiska reglering.

Sammanfattningsvis bidrar detta arbete till det framväxande området Topologisk Maskininlärning och ger värdefulla insikter för forskare inom ``transfer-inlärning'' och representationsinlärningsdomäner.
\end{scontents}
\subsection*{Nyckelord}
\begin{scontents}[store-env=keywords,print-env=true]
% SwedishKeywords were set earlier, hence we can use alternative 2
\InsertKeywords{swedish}
\end{scontents}
\end{abstract}
\babelpolyLangStop{swedish}
\cleardoublepage

\section*{Acknowledgments}
\markboth{Acknowledgments}{}
I would like to take a moment to express my heartfelt gratitude to all those who have played a pivotal role in shaping this work, representing the culmination of my academic endeavors thus far. Their unwavering support and guidance have been instrumental in this journey, and it is with sincere appreciation that I acknowledge their contributions.\\

First and foremost, I extend my deepest thanks to my supervisor, Martina Scolamiero, for her consistent support and guidance since the beginning. Martina has always been incredibly supportive, allowing me the freedom to pursue topics I am passionate about. Her valuable directions have shaped this Thesis into what it is today. I am truly grateful for the trust and support she has shown me throughout these months.\\

I am also incredibly grateful to my co-supervisor, Giovanni Luca Marchetti, whose infectious enthusiasm and invaluable ideas have pushed this project beyond my expectations. Giovanni's constant availability and support during challenging times have been invaluable. I deeply admire his passion, which permeates every aspect of his work. Collaborating with him has been an absolute privilege, and I am determined to approach future research endeavors with the same level of dedication he exemplifies.\\

Furthermore, I would like to express my sincere gratitude to Luca Moschella, the first author of the relative transformation paper, for generously providing access to the project's materials and clarifying and clarifying any uncertainties I had about the methodologies involved.\\

In addition, I wish to extend my heartfelt appreciation to Hector Barge for sparking my fascination with topology. During the most challenging moments of the pandemic lockdown, I found solace in the fascinating world he unveiled in each lecture.\\ 

Lastly, I am forever indebted to my family for their unwavering belief in me and their constant encouragement. Their love and support have been the bedrock upon which I have built my academic career. To my dear friends, I am grateful for your enduring presence and patience as you listened to my endless monologues about the usefulness of counting holes in a 768-dimensional space. Your presence in my life has been a constant source of inspiration and joy.\\

In conclusion, I would like to emphasize that the successful completion of this work would not have been possible without the unwavering support, guidance, and contributions of these exceptional individuals. Their impact on my academic journey is indelible, and for that, I offer my most profound appreciation.
\acknowlegmentssignature

\end{document}