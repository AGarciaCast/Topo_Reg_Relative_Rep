\makeatletter
\newcommand{\DeclareLatinAbbrev}[2]{%
  \DeclareRobustCommand{#1}{%
    \@ifnextchar{.}{\textit{#2}}{%
      \@ifnextchar{,}{\textit{#2.}}{%
        \@ifnextchar{!}{\textit{#2.}}{%
          \@ifnextchar{?}{\textit{#2.}}{%
            \@ifnextchar{)}{\textit{#2.}}{%
              {\textit{#2.,\ }}}}}}}}%
}
\makeatother
\DeclareLatinAbbrev{\eg}{e.g}
\DeclareLatinAbbrev{\Eg}{E.g}
\DeclareLatinAbbrev{\ie}{i.e}
\DeclareLatinAbbrev{\Ie}{I.e}
\DeclareLatinAbbrev{\etc}{etc}
\DeclareLatinAbbrev{\etal}{et~al}

\def\first {$(i)$\xspace}
\def\Second{$(ii)$\xspace}
\def\third {$(iii)$\xspace}
\def\fourth{$(iv)$\xspace}
\def\fifth {$(v)$\xspace}
\def\sixth {$(vi)$\xspace}
\def\seventh{$(vii)$\xspace}
\def\eighth{$(viii)$\xspace}

%%% custom definitions
%% Coloring the links!
\newcommand\myshade{75} % Usage: red!\myshade!black

\definecolor{ForestGreen}{RGB}{34,  139,  34}
\definecolor{HeraldRed2}{rgb}{0.81, 0.12, 0.15}

\definecolor{ffqqqq}{rgb}{1,0,0}
\definecolor{wwzzqq}{rgb}{0.4,0.6,0}
\definecolor{qqqqff}{rgb}{0,0,1}

\newcommand{\refscolor} {blue}
\newcommand{\linkscolor}{HeraldRed2}
\newcommand{\urlscolor} {ForestGreen}

%% Some definitions of used colors
%\definecolor{darkblue}{rgb}{0.0,0.0,0.3} %% define a color called darkblue
%\definecolor{darkred}{rgb}{0.4,0.0,0.0}
%\definecolor{red}{rgb}{0.7,0.0,0.0}
%\definecolor{lightgrey}{rgb}{0.8,0.8,0.8} 
%\definecolor{grey}{rgb}{0.6,0.6,0.6}
%\definecolor{darkgrey}{rgb}{0.4,0.4,0.4}
%\definecolor{aqua}{rgb}{0.0, 1.0, 1.0}

% For runin headings
\newcommand{\smartparagraph}[1]{\vspace{.05in}\noindent\textbf{#1}}

%% Table of Contents (ToC) depth 
\setcounter{secnumdepth}{4} % how many sectioning levels to assign numbers to
\setcounter{tocdepth}{4}    % how many sectioning levels to show in ToC

%% Limit hyphenation
\hyphenpenalty=9000
\tolerance=5000
% Reduce hyphenation as much as possible:
%\hyphenpenalty=15000
%\tolerance=1000

% For notes by the authors to themselves
\newcommand*{\todoinline}[1]{\textcolor{red}{TODO: #1}}

%\DeclareUnicodeCharacter{2003}{\quad}

\renewcommand{\phi}{\varphi}
\renewcommand{\epsilon}{\varepsilon}

\DeclarePairedDelimiter\abs{\lvert}{\rvert}%
\DeclarePairedDelimiter\norm{\lVert}{\rVert}%


\newtheorem*{rep@theorem}{\rep@title}
\newcommand{\newreptheorem}[2]{%
\newenvironment{rep#1}[1]{%
 \def\rep@title{#2 \ref{##1}}%
 \begin{rep@theorem}}%
 {\end{rep@theorem}}}
\makeatother


\theoremstyle{plain}% default
\newtheorem{theorem}{Theorem}[section]
\newreptheorem{theorem}{Theorem}
\newtheorem*{corollary}{Corollary}
\newtheorem{lemma}[theorem]{Lemma}
\newreptheorem{lemma}{Lemma}
\newtheorem{proposition}{Proposition}[section]
\newreptheorem{proposition}{Proposition}

\theoremstyle{definition}
\newtheorem{definition}{Definition}[section]
\newtheorem{exmp}{Example}[section]
\newtheorem{property}{Property}[section]

\theoremstyle{remark}
\newtheorem*{remark}{Remark}
\newtheorem*{note_math}{Note}

\newcommand{\bigO}{\mathcal{O}}
\newcommand{\cupdot}{\mathbin{\dot\cup}}
\DeclareMathOperator*{\argmax}{arg\,max}
\DeclareMathOperator*{\argmin}{arg\,min}
\newcommand\given[1][]{\:#1\vert\:}
\newcommand{\E}{\mathbb{E}}
\newcommand{\N}{\mathcal{N}}
\newcommand{\1}[2]{\mathds{1}_{#1, #2}}
\DeclarePairedDelimiter\floor{\lfloor}{\rfloor}
\newcommand{\conseq}{\overset{+}{=}}
